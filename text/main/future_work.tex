\section{Future Work}\label{future_work}

The road does not end here. While we have verified that ConCap is capable of generating representative traffic, further research is required to build a fully functioning ML-NIDS system around it. Below, we sketch out a number of directions further research could go into.

First, ConCap itself is limited to fully connected networks and simple "run attack command" attacks. In order to be fully representative of our modern networks, support for further network segmentation, switches, firewalls, routers... is needed. In an ideal world, a system administrator should be able to reconstruct their network within ConCap, with all of the fine details that go into configuring a company network. This system administrator should then be able to deploy attacks against the network from a library, train an NIDS system and verify the protections it provides. Furthermore, multi-stage and event-based attacks should be supported, as currently, one cannot easily run a tailored attack based on the current network traffic or conditions.

Second avenue could be a real-time ML-NIDS. The ML model itself is but a small part at the heart of an ML-NIDS system. We have not spent any time studying how to provide the NetFlows to the machine learning model on the fly, instead we post-processed the PCAP files into NetFlows and trained a model on that. While useful, this method will only detect an attack after the fact; and after the fact might be too late, certainly for attacks such as Heartbleed, where leaking the private key spells disaster and could be non-trivial to remediate. A real-time ML-NIDS could detect an attack while it is taking place, enhancing the security of the network that much more.

Third, ConCap can be used not just for its generation capabilities, but also for simulation. This thesis has shown that ConCap is a sound framework for traffic generation, therefore, we propose that this generative capability need not only be used in the context of machine learning, but could also be used as a practical tool for blue team training. Currently, the ConCap environment is spun up, attacks are executed, traffic is captured and the environment is shut down. We envision an environment that remains running, with defenders having the opportunity to train and improve their responses to various attacks.