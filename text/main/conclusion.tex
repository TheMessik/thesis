\section{Conclusion}

ML-NIDS systems are the next frontier in NIDS research. Being based on machine learning methods, these systems are capable of learning to distinguish between malicious and benign traffic without the need to manually program them with expert knowledge. These systems do rely on high quality datasets, of which there were numerous attempts to construct; each plagued with different issues. 

In this thesis, we have experimentally verified a new method of constructing ML-NIDS datasets through the ConCap framework by reconstructing the CIC-IDS-2017 dataset into ConCap scenarios and showing that the resulting dataset is representative of the original. This quality has been observed by the commonality of features that the models performed best on. Furthermore, we have further shown that the robustness of ML-NIDS models can be increased by augmenting their training with ConCap traffic, which improves their generalization performance to unseen samples from other datasets. This further cements ConCap (and other such frameworks) as the next step in ML-NIDS dataset generation and augmentation. 